\documentclass[a4paper,12pt]{article}
\usepackage{amsmath}
\usepackage{enumitem}
\usepackage{geometry}
\usepackage{esvect}
\usepackage{textcomp}
\usepackage{siunitx}
\usepackage{microtype}
\usepackage{parskip}

\begin{document}
   \null\hfill\begin{tabular}[t]{l@{}}
      \textbf{Nama: Aldzikri Dwijayanto P.} \\
      \textbf{NIM:\@ 195410189} \\
      \textbf{Kelas: TI-4}
   \end{tabular} 

   \begin{enumerate}
      \item Hasil pengukuran panjang paku adalah 9cm. jika panjang paku sesungguhnya adalah 10cm, hitung :
      \begin{enumerate}[label=\alph*.]
            \item galat
            \item Persen galat relatif
      \end{enumerate}

      Jawab
      \begin{enumerate}[label= (\alph*)]
         \item 
         $
            E_{t} = 10 - 9 = 1cm
         $\\

         \item
         $
            E_{t} = \frac{1}{10} \times 100\% \\
            = 10 \%
         $\\[2ex]
      \end{enumerate}

      \item Diberikan deret tak hingga sebagai berikut :\\
         $
            f(x) = 1 + x + \frac{x^{2}}{2!} +  \frac{x^{3}}{3!} + \frac{x^{4}}{4!} + ...
         $\\
         Hitung:
         \begin{enumerate}[label=\alph*.]
            \item 
               $
               f(x) = e ^{-\frac{1}{2}}
               $
               dengan deret diatas sebagai cacah suku = 5

            \item Jika nilai eksak $f(x) = e ^{-\frac{1}{2}} = 0,060653$, hitunglah galat dan persen galat relatif
         \end{enumerate}
         
      Jawab\\
         \begin{enumerate}[label=(\alph*)]
               \item
             \begin{flalign*}
                f(x) &= e^{x}\\
                e^{x} &= e^{-\frac{1}{2}} , x = -\frac{1}{2} = 0,5\\
                f(-\frac{1}{2}) &=
             1+(-\frac{1}{2})+\frac{(-\frac{1}{2})^{2}}{2!}+\frac{(-\frac{1}{2})^{3}}{3!}+\frac{(-\frac{1}{2})^{4}}{4!}\\
                &=1+(-0,5)+\frac{0,25}{2}+(\frac{-0,125}{6})+\frac{0,0625}{24}\\
                &=1-0,5+0,125-0,02083+0,002604\\
                &=0,60677
             \end{flalign*}

               \item 
            \begin{flalign*}
               galat \implies E_{t} &= 0,60653 - 0,60677\\
               &=-0,00024
            \end{flalign*}

               \begin{flalign*}
                  Persen\ galat \implies E_{t}&=\frac{-0,00024}{0,60653} \times 100\%\\
                  &=0,039
               \end{flalign*}
         \end{enumerate}

   \end{enumerate}
\end{document}
