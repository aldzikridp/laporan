% !TeX spellcheck = id_ID
\documentclass[a4paper,12pt]{article}
\usepackage[bahasa]{babel}
\usepackage{graphicx}
\usepackage{multirow}
\usepackage{enumitem}
\usepackage{listings}
\usepackage{wrapfig}
\usepackage[T1]{fontenc}
\usepackage{inconsolata}
\usepackage{lipsum}

\usepackage{color}
\usepackage[table]{xcolor}
\definecolor{pblue}{rgb}{0.13,0.13,1}
\definecolor{pgreen}{rgb}{0,0.5,0}
\definecolor{pred}{rgb}{0.9,0,0}
\definecolor{pgrey}{rgb}{0.46,0.45,0.48}
\lstset{language=Java,
	showspaces=false,
	showtabs=false,
	breaklines=true,
	showstringspaces=false,
	breakatwhitespace=true,
	commentstyle=\color{pgreen},
	keywordstyle=\color{pblue},
	stringstyle=\color{pred},
	rulecolor=\color{black},
	basicstyle=\ttfamily,
	moredelim=[il][\textcolor{pgrey}]{$$},
	moredelim=[is][\textcolor{pgrey}]{\%\%}{\%\%}
}

\graphicspath{ {./img/} }
\begin{document}
\title{ {\Large Laporan Praktikum}\\ Algoritma dan Pemrograman \\{\Large Pertemuan 8}}

\author{Aldzikri Dwijayanto Prathama 
	\\195410189
	\\Teknik Informatika}
\makeatletter
\begin{titlepage}
	\begin{center}
		{\huge \bfseries \@title }\\[14ex]
		\includegraphics[scale=.8]{logo}\\[4ex]
		{\large \@author}\\[19ex]
		{\large \bfseries {SEKOLAH TINGGI MANAJEMEN INFORMATIKA DAN KOMPUTER
				AKAKOM YOGYAKARTA}}
	\end{center}


%{\large \@date} 
\end{titlepage}
\makeatother
%\maketitle
\newpage
\tableofcontents
\newpage

\section{Tujuan}
Mahasiswa dapat mengimplementasikan menyelesaikan kasus konsep perulangan while untuk
\section{Dasar Teori}
\paragraph{}
Perulangan WHILE banyak digunakan pada program yang terstruktur. Perulangan
ini banyak digunakan bila jumlah perulangannya belum diketahui. Proses perulangan
akan terus berlanjut selama kondisinya bernilai benar (true) dan akan berhenti bila
kondisinya bernilai salah.\\
Karakteristik while() adalah:
\begin{enumerate}
	\item Dilakukan pengecekan kondisi terlebih dahulu sebelum dilakukan perulangan. Jika kondisi yang dicek bernilai benar (true) maka perulangan akan dilakukan.
	\item Blok statement tidak harus ada. Struktur tanpa statement akan tetap dilakukan
	selama kondisi masih true.
\end{enumerate}
Bentuk umum :\\

\begin{lstlisting}[frame=single, linewidth=5cm]
While(ungkapan)
	Pernyataan
\end{lstlisting} 

Keterangan :
\begin{itemize}
	\item bagian pernyataan akan diekseskusi selama ungkapan dalam while bernilai benar.
	\item Pengujian terhadap ungkapan pada while dilakukan sebelum bagian pernyataan.
	\item Kemungkinan pernyataan pada while tidak dijalankan sama sekali, jika ketemu
	kondisi yang pertama kali bernilai salah.
\end{itemize}
\newpage
Flowchart perulangan while dapat dilihat seperti gambar berikut :\\
\begin{center}
	\includegraphics[scale=.5]{image--022}
\end{center}
Catatan:\\
Pernyataan perulangan dengan while akan selalu dikerjakan jika ungkapan selalu benar.
Oleh karena itu, kita harus membuat kondisi suatu saat ungkapan bernilai salah agar
perulangan berakhir.

\section{Praktik}
\subsection{Praktik 1}
\paragraph{Masalah\\}
Ketik program di bawah
\begin{lstlisting}[frame=single]
import java.util.Scanner;
public class UlangWhile1
{
    public static void main(String args[])
    {
        Scanner masuk = new Scanner(System.in);
        int bil;
        bil=1;
        while (bil<=5) {
            System.out.println(bil);
            bil=bil+1;
        }
    }
}
\end{lstlisting}
Buat flowchart untuk program diatas seperti berikut :\\
\begin{center}
	\includegraphics[scale=.5]{image--024}
\end{center}
\paragraph{Penyelesaian\\}
\begin{center}
\includegraphics[width=\linewidth]{Capture1}\\
\end{center}
\newpage
\begin{center}
 \includegraphics[scale=1]{Capture2}
\end{center}
Pada output di atas terlihat bahwa tersebut melakukan penghitungan 1 - 5, di dalam program terdapat
\begin{lstlisting}[frame=single]
while (bil<=5) {
            System.out.println(bil);
            bil=bil+1;
        }
\end{lstlisting}
yang mana fungsi \texttt{while} tersebut akan melakukan perulangan jika variabel \texttt{bil} memiliki nilai kurang dari sama dengan 5. Di dalam while terdapat pernyataan untuk mengeprint variabel \texttt{bil}, yang sebelumnya sudah diberi nilai 1, lalu akan menambahkan variabel \texttt{bil} dengan 1 sampai variabel \texttt{bil} bernilai lebih dari sama dengan 5. Maka fungsi while tersebut akan melakukan penghitungan 1 - 5.
\begin{center}
	\includegraphics[scale=.5]{image--024}
\end{center}
Dari flowchart di atas terlihat jelas alur dari program sebelumnya, fungsi \texttt{while} akan melakukan pengecekan terlebih dahulu terhadap kondisi yang di berikan, pada kasus ini yaitu bil<=5, jika benar program akan mengeluarkan nilai dari variabel \texttt{bil}, dan menambahkannya dengan satu, lalu akan mengulanginya lagi sampai kondisi yang dicek false.

\subsection{Praktik 2}
\paragraph{Masalah\\}
Modifikasi praktik 1 dengan mengubah perrnyataan bil=1 yang ada pada baris 8
menjadi bil=5, dan pernyataan while(bil<=5) yang ada dibaris ke 9 dengan
while(bil>=1) dan bil=bil+1 pada baris 11 menjadi bil=bil-1, amati hasil outputnya,
kenapa bisa demikian, jelaskan !

\paragraph{Penyelesaian\\}
\begin{center}
	\includegraphics[width=\linewidth]{Capture4}
	\includegraphics[scale=1]{Capture5}
\end{center}
Program menjadi menghitung mundur, itu karena nilai awal variabel \texttt{bil} adalah 5, lalu kondisi pada \texttt{while} adalah bil>=1 dengan salah satu pernyataannya \texttt{bil=bil+1}, itu berarti while akan mengurangi variabel \texttt{bil} yang bernilai 5 dengan 1 di setiap pengulangannya sampai variabel \texttt{bil} bernilai kurang dari sama dengan 1. 

\subsection{Praktik 3}
\paragraph{Masalah\\}
Buat program untuk menampilkan tulisan STMIK AKAKOM dan buat suatu
pernyataan jika tulisan tersebut bisa ditampilkan selama jawaban True (Ya) dan akan
di hitung jumlah yang di tampilkan
\begin{lstlisting}[frame=single, basicstyle=\small]
import java.util.Scanner;
public class modul8_3 {
    public static void main(String[] args) {
        boolean running = true;
        int counter = 0;
        String jawab;
        Scanner scan = new Scanner(System.in);
        while( running ) {
            System.out.println("STMIK AKAKOM");
            System.out.print("Tampilkan Tulisan lagi [ya/tidak]> ");
            jawab = scan.nextLine();
            // cek jawabnnya, kalau ya maka berhenti mengulang
            if( jawab.equalsIgnoreCase("tidak") ){
                running = false;
            }
                counter++;
        }
        System.out.println("Anda sudah melakukan perulangan sebanyak " + counter + " kali");
    }
}
\end{lstlisting}
\begin{enumerate}[label=\alph*.]
	\item Simpan dan jalankan
	\item Ujilah dengan mengisi ya sebanyak 2 kali , amati hasilnya
	\item Lanjutkan menguji dengan mengisi tidak, amati hasilnya
\end{enumerate}

\paragraph{Penyelesaian\\}
\begin{center}
	\includegraphics[scale=.7]{Capture6}
	\includegraphics[scale=1]{Capture7}
\end{center}
Program diatas memiliki fungsi \texttt{while} yang memiliki kondisi dari variabel running, yang sebelumnya sudah dideklarasikan dalam bentuk boolean dan memiliki nilai true. Lalu di dalam fungsi \texttt{while} terdapat seleksi if
\begin{lstlisting}[frame=single]
if( jawab.equalsIgnoreCase("tidak") ){
                running = false;
}
\end{lstlisting}
jika variabel \texttt{jawab} berisi "tidak", maka variabel "running" akan dirubah nilainya menjadi false, sehingga tidak cocok dengan kondisi di \texttt{while} dan pengulanganpun akan berhenti.

\subsection{Praktik 4}
\paragraph{Masalah\\}
Buat program dengan while untuk mencetak bilangan genap dari 0 sampai 10

\paragraph{Penyelesaian\\}
	\includegraphics[scale=.7]{Capture8}
	\includegraphics[scale=1]{Capture9}

\begin{lstlisting}[frame=single]
public class modul8_4
{
    public static void main(String args[])
    {
        int count = 2;
        while(count <= 10)
        {
            if(count == 10)
            {
                System.out.print(count);
            }
            else
            {
                System.out.print(count +", ");
            }
            count=count+2;
        }
        System.out.println("");
    }
}
\end{lstlisting}
Fungsi \texttt{while} pada program di atas memiliki kondisi count <= 10, yang berarti pengulangan akan berhenti jika variabel "count" memiliki nilai lebih besar sama dengan 10. Didalam while terdapat pernyataan seleksi, jika variabel "count" memiliki nilai 10, maka program hanya akan mengeluarkan nilai dari variabel "count". Selain dari itu maka program akan mengeluarkan nilai variabel "count" ditambah koma dan spasi (", "). Setelah itu terdapat \texttt{count=count+2}, karena yang akan dikeluarkan pada layar adalah bilangan genap, maka variabel "count" akan ditambah dengan  2 di setiap terjadi pengulangan. Lalu pada baris program terakhir terdapat \texttt{System.out.println("");}, yang akan memindahkan ke baris selanjutnya pada command line setelah program selesai dijalankan.

\section{Latihan}
\paragraph{Masalah\\}
Modifikasi praktik 4 agar bilangan genap yang dicetak dimulai dan diakhiri menurut
keinginan user.

\paragraph{Penyelesaian\\}
\begin{center}
	\includegraphics[scale=.7]{Capture10}\\
	\includegraphics[scale=1]{Capture11}
\end{center}

\begin{lstlisting}[frame=single]
import java.util.Scanner;
public class latihan8
{
    public static void main(String args[])
    {
        int count = 2;
        int batas;
        Scanner in = new Scanner(System.in);
        count = in.nextInt();
        batas = in.nextInt();
        while(count <= batas)
        {
	    if(count == batas)
            {
	            System.out.print(count);
            }
            else
	    {
	            System.out.print(count +", ");
            }
            count=count+2;
        }
        System.out.println("");
    }
}
\end{lstlisting}
Untuk program di atas, ditambahkan fungsi \texttt{Scanner} dan variabel "batas", lalu variabel "count" dan "batas" dirubah agar dapat diinpit saat program dijalankan. Pada while yang semula memiliki kondisi \texttt{count <= 10} dirubah menjadi \texttt{count <= batas}, yang berarti while akan mengulang selama variabel "count" lebih kecil sama dengan dari variabel "batas".\\
Pada pernyataan \texttt{while}, seleksi if kondisinya diganti menjadi \texttt{count == batas}, jika variabel "count" bernilai sama dengan variabel "batas", maka program hanya akan mengeluarkan nilai dari count saja di layar.

\section{Tugas}
\subsection{Tugas 1}
\paragraph{Masalah\\}
Hitunglah jumlah dan rata-rata bilangan positif, dimana banyaknya data ditentukan
dari data yang dimasukan dengan keyboard, contoh hasil output yang diinginkan
sbb:\\[1ex]
Banyaknya data : 4\\
Data ke-1 : 3\\
Data ke-2 : 5\\
Data ke-3 : 2\\
Data ke-4 : 6\\
Rata-rata : 4.0\\
Jumlah : 16.0\\
Press any key to continue . . .\\[1ex]

\paragraph{Penyelesaian\\}
\begin{center}
	\includegraphics[scale=.5]{tugas8-1}
	\includegraphics[scale=.8]{tugas8-1-2}
\end{center}
\begin{lstlisting}[frame=single]
import java.util.Scanner;
public class Tugas8_1
{
    public static void main(String args[])
    {
        Scanner in = new Scanner(System.in);
        int panjang;
        int i = 0;
        float jumlah = 0;
        float rata;
        System.out.print("Panjang data = ");
        panjang = in.nextInt();
        int[] data = new int[panjang];
        while(i <= panjang - 1)
        {
            System.out.print("Data"+i+" = ");
            data[i] = in.nextInt();
            jumlah = jumlah + data[i];
            i=i+1;
        }
        rata = jumlah/panjang;
        System.out.println("Jumlah = "+jumlah);
        System.out.println("rata-rata = "+rata);
    }
}
\end{lstlisting}

Program di atas memiliki variabel \texttt{data} yang berbentuk array integer, yang panjangnya ditentukan oleh variabel panjang. Karena array di java bersifat tetap, maka digunakan ArrayList, terlihat pada baris kode
\begin{lstlisting}
int[] data = new int[panjang];
\end{lstlisting}   
baris ini akan merubah panjang variabel \texttt{data}, dengan nilai dari variabel \texttt{panjang}.\\
Pada fungsi \texttt{while} diberi kondisi i <= panjang - 1, itu karena array dimulai dari 0, jadi misalnya kita memberi panjang array itu 4, maka akan terdiri dari 0, 1, 2, 3.\\
Di dalam \texttt{while} terdapat pernyataan
\begin{lstlisting}
data[i] = in.nextInt();
\end{lstlisting}
pernyataan ini akan menerima masukan dan memasukkannya ke variabel \texttt{data} ke "i", jadi variabel \texttt{data} akan dapat dimasukkan nilai sesuai dengan panjang array yang telah ditentukan.\\
Lalu baris pernyataan
\begin{lstlisting}
jumlah = jumlah + data[i];
\end{lstlisting}
berfungsi untuk menambahkan variabel \texttt{jumlah}, dengan nilai variabel \texttt{data} ke "i", jadi variabel \texttt{jumlah} akan bernilai jumlah semua angka yang ada di variabel \texttt{data}.\\
Setelah perulangan while selesai, program akan menjalankan pernyataan
\begin{lstlisting}
rata = jumlah/panjang;
\end{lstlisting}
yang akan memberi nilai ke variabel \texttt{rata} dari hasil operasi pembagian jumlah dengan panjang, yang menghasilkan rata-rata dari semua nilai di variabel \texttt{data}

\end{document}