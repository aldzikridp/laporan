\documentclass[a4paper,12pt]{article}
\usepackage[indonesian]{babel}
\usepackage{graphicx}
\usepackage{multirow}
\usepackage{enumitem}
\usepackage{listings}
\usepackage{wrapfig}
\usepackage[T1]{fontenc}
\usepackage{inconsolata}
\usepackage{lipsum}
\usepackage{adjustbox}


\usepackage{color}
\usepackage[table]{xcolor}
\definecolor{mygreen}{rgb}{0,0.6,0}
\definecolor{mygray}{rgb}{0.5,0.5,0.5}
\definecolor{mymauve}{rgb}{0.58,0,0.82}
\lstset{%
    language=java,
    showstringspaces=false,          % Prevent tex replacing space to bracket in code
    frame=single,                    % Set frame around code
    backgroundcolor=\color{white},   % choose the background color
    basicstyle=\footnotesize,        % size of fonts used for the code
    breaklines=true,                 % automatic line breaking only at whitespace
    captionpos=b,                    % sets the caption-position to bottom
    commentstyle=\color{mygreen},    % comment style
    keywordstyle=\color{blue},       % keyword style
    stringstyle=\color{mymauve},     % string literal style
}

\graphicspath{ {./img/} }
\begin{document}
\title{ {\Large Laporan Praktikum}\\ Algoritma dan Pemrograman Lanjut\\{\Large Pertemuan 5}}

\author{Aldzikri Dwijayanto Prathama
    \\195410189
    \\Informatika}
\makeatletter
\begin{titlepage}
    \begin{center}
        {\huge \bfseries \@title}\\[14ex]
        \includegraphics[scale=.8]{logo}\\[4ex]
        {\large \@author}\\[12ex]
        {\large \bfseries {SEKOLAH TINGGI MANAJEMEN INFORMATIKA DAN KOMPUTER
            AKAKOM YOGYAKARTA}}
    \end{center}


%{\large \@date}
\end{titlepage}
\makeatother
%\maketitle
\newpage
\tableofcontents
\newpage

\section{Tujuan}
\paragraph{}
Mahasiswa dapat:
\begin{enumerate}
    \item Menjelaskan konsep array 3 dimensi
    \item Merencanakan struktur data dalam bentuk array 3 dimensi
    \item Mengaplikasikan array 3 dimensi
\end{enumerate}


\section{Teori}
\paragraph{}
Array 3 dimensi adalah array yang tidak jauh berbeda dari array
dimensi satu dan dua yang telah dijelaskan sebelumnya, kecuali pada indeks
dari array. Pada tipe ruang misalnya type ruang = array [1..8,1..5,1..3] of
integer; menunjukkan bahwa ruang adalah nama-pengenal/variabel yang
berupa array yang komponennya bertipe integer dan terdiri atas 8 baris,
mempunyai 5 kolom dan 3 halaman.

\newpage

\section{Pembahasan}
\subsection{Praktik}
\subsubsection{Praktik 1}

\end{document}
