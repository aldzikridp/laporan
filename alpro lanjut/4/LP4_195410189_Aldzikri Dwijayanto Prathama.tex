% !TeX spellcheck = id_ID
\documentclass[a4paper,12pt]{article}
\usepackage[indonesian]{babel}
\usepackage{graphicx}
\usepackage{multirow}
\usepackage{enumitem}
\usepackage{listings}
\usepackage{wrapfig}
\usepackage[T1]{fontenc}
\usepackage{inconsolata}
\usepackage{lipsum}
\usepackage{adjustbox}


\usepackage{color}
\usepackage[table]{xcolor}
\definecolor{mygreen}{rgb}{0,0.6,0}
\definecolor{mygray}{rgb}{0.5,0.5,0.5}
\definecolor{mymauve}{rgb}{0.58,0,0.82}
\lstset{%
    language=java,
    showstringspaces=false,          % Prevent tex replacing space to bracket in code
    frame=single,                    % Set frame around code
    backgroundcolor=\color{white},   % choose the background color
    basicstyle=\footnotesize,        % size of fonts used for the code
    breaklines=true,                 % automatic line breaking only at whitespace
    captionpos=b,                    % sets the caption-position to bottom
    commentstyle=\color{mygreen},    % comment style
    escapeinside={\%*}{*)},          % if you want to add LaTeX within your code
    keywordstyle=\color{blue},       % keyword style
    stringstyle=\color{mymauve},     % string literal style
}

\graphicspath{ {./img/} }
\begin{document}
\title{ {\Large Laporan Praktikum}\\ Algoritma dan Pemrograman Lanjut\\{\Large Pertemuan 4}}

\author{Aldzikri Dwijayanto Prathama 
	\\195410189
	\\Informatika}
\makeatletter
\begin{titlepage}
	\begin{center}
		{\huge \bfseries \@title }\\[14ex]
		\includegraphics[scale=.8]{logo}\\[4ex]
		{\large \@author}\\[12ex]
		{\large \bfseries {SEKOLAH TINGGI MANAJEMEN INFORMATIKA DAN KOMPUTER
				AKAKOM YOGYAKARTA}}
	\end{center}


%{\large \@date} 
\end{titlepage}
\makeatother
%\maketitle
\newpage
\tableofcontents
\newpage

\section{Tujuan}
\paragraph{}
Mahasiswa dapat :
\begin{enumerate}
    \item Menjelaskan konsep array 2 dimensi
    \item Merencanakan struktur data dalam bentuk array 2 dimensi
    \item Mengaplikasikan array 2 dimens
\end{enumerate}

\section{Teori}
\paragraph{}
Array 2 Dimensi atau bisa disebut juga Array Multi Dimensi ,adalah versi lanjut
dari Array biasa ,yang merupakan sebuah deretan atau susunan , nama-nama
variable( element) , yang memiliki tipe data sama dalam struktur list atau daftar,
27yang dapat diakses secara baris dan kolom, berdasarkan element/indexnya.

\newpage

\section{Pembahasan}
\subsection{Praktik}
\subsubsection{Praktik 1}
\begin{lstlisting}
public class Array2 {
    public static void main(String[] args) {
        String cats[][]= {{"terry","brown"},
        {"kitty","white"},
        {"toby","gray"},
        {"fido","black"}};
        System.out.println("Nama Kucing\tWarna");
        System.out.println(cats[0][0] + "\t\t" + cats[0][1]);
        System.out.println(cats[1][0] + "\t\t" + cats[1][1]);
        System.out.println(cats[2][0] + "\t\t" + cats[2][1]);
        System.out.println(cats[3][0] + "\t\t" + cats[3][1]);
    }
}
\end{lstlisting}
Program praktik 1 tersebut memiliki matriks 2 dimensi yang memiliki ukuran 4 baris dan 2 kolom (4x2). Matriks tersebut kemudian diisi dengan nama kucing untuk 
kolom satu, dan warna dari kucing tersebut pada kolom kedua.
\begin{center}
    \includegraphics{1.png}
\end{center}

\subsubsection{Praktik 2}
\begin{lstlisting}
public class Array2a {
    public static void main(String[] args) {
        String cats[][]= {{"terry","brown"},
        {"kitty","white"},
        {"toby","gray"},
        {"fido","black"}};
        System.out.println("Nama Kucing\tWarna");
        for (int i=0;i<cats.length;i++) {
            for (int j=0;j<cats[i].length;j++) {
                System.out.print(cats[i][j]);
                System.out.print("\t");
            }
            System.out.println(" ");
        }
    }
}
\end{lstlisting}
Pada praktik 2, adalah mengubah program pada praktik 1, yang semula cara mengeprintnya secara manual, pada praktik 2 dimodifikasi agar mengeprint menggunakan 
perulangan. Karena arraynya mreupakan array dua dimensi, maka digunakan dua perulangan. Yang perulangan pertama berguna untuk menghitung baris, dan perulangan 
kedua untuk menghitung kolom.
\begin{center}
    \includegraphics{2.png}
\end{center}

\subsubsection{Praktik 3}
\begin{lstlisting}
import java.util.Scanner;
public class Array2b {
    public static void main(String[] args) {
        String cats[][] = new String[4][2];
        Scanner in = new Scanner(System.in);
        for (int i=0;i<cats.length;i++) {
            for (int j=0;j<cats[i].length;j++) {
                cats[i][j] = in.nextLine();
            }
        }
        System.out.println("Nama Kucing\tWarna");
        for (int i=0;i<cats.length;i++) {
            for (int j=0;j<cats[i].length;j++) {
                System.out.print(cats[i][j]);
                System.out.print("\t");
            }
            System.out.println(" ");
        }
    }
}
\end{lstlisting}
Program tersebut merupakan program dari praktik 2, yang dimodifikasi sehingga mampu menerima input dari user. Perulangan yang digunakan untuk mneginputkan
sama dengan perulangan yang sebelumnya digunakan untuk mengeprint array, sehingga input pertama akan memasukkan nama kucing, dan 
input kedua akan memasukkna warna kucing.
\begin{center}
    \includegraphics{3.png}
\end{center}

\subsubsection{Praktik 4}
\begin{lstlisting}
import java.util.Scanner;
public class Matrik {
    public static void main(String[] args) {
        Scanner input = new Scanner(System.in);
        int[][] x = {{1, 2, 3}, {4, 5, 6}};
        int[][] y = {{3, 6, 1}, {4, 7, 9}};
        int baris = 2;
        int kolom = 3;
        int[][] z = new int[baris][kolom];
        System.out.println("ini adalah matrix x");
        for (int i = 0; i < baris; i++) {
            for (int j = 0; j < kolom; j++) {
                System.out.print(x[i][j] + " ");
            }
            System.out.println();
        }
        System.out.println("ini adalah matrix y");
        for (int i = 0; i < baris; i++) {
            for (int j = 0; j < kolom; j++) {
                System.out.print(y[i][j] + " ");
            }
            System.out.println();
        }
    }
}
\end{lstlisting}
Pada praktik 4 program memiliki dua array yang berisi matrix. Kemudian kedua matriks tersebut diprint menggunakan perulangan seperti program pada praktik 2.
\begin{center}
    \includegraphics{4.png}
\end{center}

\subsubsection{Praktik 5}
\begin{lstlisting}
import java.util.Scanner;
public class Matrika {
    public static void main(String[] args) {
        Scanner input = new Scanner(System.in);
        int[][] x = {{1, 2, 3}, {4, 5, 6}};
        int[][] y = {{3, 6, 1}, {4, 7, 9}};
        int baris = 2;
        int kolom = 3;
        int[][] z = new int[baris][kolom];
        System.out.println("ini adalah matrix x");
        for (int i = 0; i < baris; i++) {
            for (int j = 0; j < kolom; j++) {
                System.out.print(x[i][j] + " ");
            }
            System.out.println();
        }
        System.out.println("ini adalah matrix y");
        for (int i = 0; i < baris; i++) {
            for (int j = 0; j < kolom; j++) {
                System.out.print(y[i][j] + " ");
            }
            System.out.println();
        }
        System.out.println("hasil dari x-y");
        for (int i = 0; i < baris; i++) {
            for (int j = 0; j < kolom; j++) {
                z[i][j]=x[i][j]-y[i][j];
                System.out.print(z[i][j]+" ");
            }
            System.out.println();
        }
        System.out.println("hasil dari x+y");
        for (int i = 0; i < baris; i++) {
            for (int j = 0; j < kolom; j++) {
                z[i][j]=x[i][j]+y[i][j];
                System.out.print(z[i][j]+" ");
            }
            System.out.println();
        }
    }
}
\end{lstlisting}
Praktik 5 adalah memodifikasi dari praktik 4, sehingga bisa melakukan operasi penambahan dan pengurangan pada matrix yang di masukkan. Pada perulangan 
pertama dan kedua program akan menginputkan matriks pertama dan kedua menginputkan matriks. Perulangan kedua akan mengurangi matriks x dengan matriks y, 
dengan mengurangi baris dengan baris, dan kolom dengan kolom, kemudian mengeprintnya. Kemudian perulangan ketiga akan menambahkan matriks x dengan matriks y, baris dengan baris, kolom dengan kolom, kemudian mengeprintnya.
\begin{center}
    \includegraphics{5.png}
\end{center}

\subsubsection{Praktik 6}
\begin{lstlisting}
public class MatriksTranspose {
    public static void main(String[] args) {
        int[][] matriks =
        {{12,23,32},{34,56,63},{78,89,97}};
        int j,k;
        System.out.println("Matriks Sebelum Transpose");
        for(j=0;j<3;j++){
            for(k=0;k<3;k++){
                System.out.print(matriks[j][k]+" ");
            }
            System.out.println();
        }
        System.out.println("\nMatriks Setelah Transpose");
        for(j=0;j<3;j++){
            for(k=0;k<3;k++){
                System.out.print(matriks[k][j]+" ");
            }
            System.out.println();
        }
    }
}
\end{lstlisting}
Program praktik 6 akan menjalankan operasi transpose terhadap array matriks. Untuk melakukannya digunakan perulangan, yang semula perulangan pertama untuk 
menhitung baris, dan perulangan kedua untuk kolom, dibalik sehingga perulangan pertama untuk menghitung kolom, dan perulangan kedua untuk menghitung baris.
\begin{center}
    \includegraphics{6.png}
\end{center}

\end{document}
