% !TeX spellcheck = id_ID
\documentclass[a4paper,12pt]{article}
\usepackage[bahasa]{babel}
\usepackage{graphicx}
\usepackage{multirow}
\usepackage{enumitem}
\usepackage{listings}
\graphicspath{ {./img/} }
\begin{document}
\title{Laporan Praktikum Statistika Pertemuan 7}
%\author{Aldzikri Dwijayanto Prathama \\ {\small 195410189}}
\author{Aldzikri Dwijayanto Prathama 
	\\195410189}
\makeatletter
\begin{titlepage}
	\begin{center}
		{\huge \bfseries \@title }\\[14ex]
		\includegraphics[scale=.8]{logo}\\[4ex]
		{\large \@author}\\[20ex]
		{\large \bfseries {SEKOLAH TINGGI MANAJEMEN INFORMATIKA DAN KOMPUTER
				AKAKOM YOGYAKARTA}}
	\end{center}


%{\large \@date} 
\end{titlepage}
\makeatother
%\maketitle
\newpage
\tableofcontents
\newpage
\section{Tujuan}
\begin{enumerate}
	\item Praktikan dapat melakukan penyajian data dalam bentuk tabel Kontingensi
	\item Praktikan dapat melakukan penyajian data dalam bentuk table distribusi Frekuensi
\end{enumerate}
\section{Dasar Teori}
\subsection{Tabel Kontigensi}
\paragraph{}
Tabel Kontigensi merupakan tabel yang digunakan untuk mengukur hubungan (asosiasi) antara dua variable kategorik dimana tabel tersebut merangkum frekuensi bersama dari observasi pada setiap kategori variable. 
\paragraph{}
Misalkan n sampel diklasifikasikan secara silang berdasarkan dua atribut atau lebih dalam suatu.
\paragraph{}
Berikut ini contoh tabel kontingesi 2x2 :
\begin{table}[!ht]
	\begin{tabular}{|c|c|l|l|c|}
		\hline
		\multicolumn{2}{|l|}{\multirow{2}{*}{}} & \multicolumn{2}{l|}{Variable 2}                 & \multirow{2}{*}{Total} \\ \cline{3-4}
		\multicolumn{2}{|l|}{}                  & \multicolumn{1}{c|}{1} & \multicolumn{1}{c|}{2} &                        \\ \hline
		\multirow{2}{*}{Variable 1}     & 1     & $O_{11}$                       &$O_{12}$                      &\multicolumn{1}{l|}{$n_{1+}$ }  \\ \cline{2-5} 
		& 2     & $O_{21}$                       & $O_{22}$                       & \multicolumn{1}{l|}{$n_{2+}$}  \\ \hline
		\multicolumn{2}{|c|}{Total}             & $n_{+1}$                       &  $n_{+2}$                      & N                      \\ \hline
	\end{tabular}
\end{table}
\paragraph{}
Dengan menggunakan R Console maka :
\begin{enumerate}
	\item Untuk membuat Tabel \textit{Contingency} dua arah dengan Fungsi \textit{\textbf{table( )}} dari  \textit{data.frame}
	\item Untuk membuat Tabel Contingency tiga Arah atau lebih dengan Fungsi \textit{\textbf{ftable( )}} untuk membentuk tabel contingency tiga arah dari \textit{data.frame}.
\end{enumerate}
\subsection{Distribusi Frekuensi}
\paragraph{}
Distribusi frekuensi adalah daftar nilai data (bisa nilai individual atau nilai data yang sudah dikelompokkan ke dalam selang interval tertentu) yang disertai dengan nilai frekuensi yang sesuai.
\paragraph{}
Pengelompokkan data ke dalam beberapa kelas dimaksudkan agar ciri-ciri penting data tersebut dapat segera terlihat. Daftar frekuensi ini akan memberikan gambaran yang khas tentang bagaimana keragaman data.
\paragraph{}
Distribusi frekuensi dibuat dengan alasan berikut:
\begin{enumerate}[label=\alph*.]
	\item kumpulan data yang besar dapat diringkas
	\item kita dapat memperoleh beberapa gambaran mengenai karakteristik data, dan
	\item merupakan dasar dalam pembuatan grafik penting (seperti histogram).
\end{enumerate}
\paragraph{}
Pada saat menyusun tabel distribusi frekuensi, pastikan bahwa 
\begin{enumerate}[label=\alph*.]
	\item kelas tidak tumpang tindih sehingga setiap nilai-nilai pengamatan harus masuk tepat ke dalam satu kelas
	\item tidak akan ada data pengamatan yang tertinggal (tidak dapat dimasukkan ke dalam kelas tertentu). 
\end{enumerate}
\paragraph{}
Dengan menggunakan R Console maka :
\begin{enumerate}
	\item Penyajian data dalam bentuk tabel  distribusi frequensi dapat digunakan \textbf{fungsi \textit{table()}}
	\item Untuk  penyajian data dalam bentuk tabel  distribusi frequensi relatif  digunakan \textbf{fungsi \textit{table()/length()}}
	\item Untuk Membuat Tabel Distribusi Frekuensi untuk Data Berkelompok digunakan \textbf{fungsi \textit{cut( )}} untuk membuat suatu interval. Argumen \textbf{\textit{break}} digunakan untuk menentukan batas-batas interval.	
\end{enumerate}
\section{Praktik}
\subsection{Tabel Kontigensi}
\subsubsection{Praktik 1}
\paragraph{Soal\\}
Sajikan data berikut ini dalam bentuk table kontigensi 
\begin{table}[!ht]
	\begin{tabular}{|c|c|}
		\hline 
		Pendidikan & Jenis Kelamin \\ 
		\hline 
		S1 & Laki-laki \\ 
		\hline 
		S1 & Laki-laki \\ 
		\hline 
		S1 & Laki-laki \\ 
		\hline 
		S1 & Perempuan \\ 
		\hline 
		S1 & Perempuan \\ 
		\hline 
		S2 & Perempuan \\ 
		\hline 
		S2 & Perempuan \\ 
		\hline 
		S2 & Perempuan \\ 
		\hline 
		S2 & Perempuan \\ 
		\hline 
		S2 & Laki-laki \\ 
		\hline 
	\end{tabular}
\end{table} 
\paragraph{Penyelesaian\\}
Buat variabel dan masukkan data dari variabel tersebut dengan dengan fungsi c dengan format perintah\\ 
\texttt{>variabel <- c("data 1","data 2"..."data n").}\\
\includegraphics[width=\linewidth]{1}\\
Pada praktik ini dibuat variabel \texttt{pendidikan}, yang berisi data jenjang pendidikan, dan variabel \texttt{jenis\_kelamin} yang berisi jenis kelamin.\\
Seteleh itu dibuat tabel kontigensi dua arah dari variabel \texttt{data\_frame}, dengan perintah\\
\texttt{>table(data\_frame)}
\subsubsection{Praktik 2}
\paragraph{Soal\\}
Sajikan data berikut ini dalam tabel kontigensi
\begin{table}[!ht]
	\begin{tabular}{|l|l|l|l|l|}
		\hline
		No & Jenis\_kelamin & \multicolumn{1}{c|}{Pendidikan} & \multicolumn{1}{c|}{status} & \multicolumn{1}{c|}{hobi} \\ \hline
		1  & Laki-laki      & S1                              & Sudah menikah               & membaca                   \\ \hline
		2  & Laki-laki      & S1                              & Sudah menikah               & membaca                   \\ \hline
		3  & Laki-laki      & S1                              & Belum menikah               & membaca                   \\ \hline
		4  & Perempuan      & S1                              & Sudah menikah               & membaca                   \\ \hline
		5  & Perempuan      & S1                              & Sudah menikah               & memasak                   \\ \hline
		6  & Perempuan      & S2                              & Belum menikah               & membaca                   \\ \hline
		7  & Perempuan      & S2                              & Sudah menikah               & membaca                   \\ \hline
		8  & Perempuan      & S2                              & Belum menikah               & memasak                   \\ \hline
		9  & Perempuan      & S2                              & Belum menikah               & membaca                   \\ \hline
		10 & Laki-laki      & S2                              & Sudah menikah               & memasak                   \\ \hline
	\end{tabular}
\end{table}
\paragraph{Penyelesaian\\}
Buat variabel dan masukkan data dari variabel tersebut dengan format perintah\\ \texttt{>variabel <- c("data 1","data 2"..."data n").}\\
\includegraphics[width=\linewidth]{2}\\
Untuk praktek ini buat variabel \texttt{pendidikan} dan masukkan data dari pendidikan ke variabel tersebut dengan fungsi \texttt{c}\\
Selanjutnya buat variabel \texttt{hobi}, \texttt{jenis\_kelamin}, dan \texttt{status} lalu beri data sesuai klasifikasi variabel tersebut.\\
Lalu buat data frame yang terdiri dari variabel \texttt{jenis\_kelamin}, \texttt{pendidikan}, \texttt{status}, \texttt{hobi} dengan nama data\_frame,dengan perintah\\ 
\texttt{>data\_frame <- data.frame(jenis\_kelamin,pendidikan,status,hobi)}\\
lalu buat tabel kontigensi dari tabel data\_frame, karena lebih dari dua arah maka tidak menggunakan perintah \texttt{table}, gunakan perintah\\
\texttt{>ftable(data\_frame)}

\subsection{Distribusi Frekuensi}
\subsubsection{Praktik 1}
\paragraph{Soal\\}
Dari data berikutini sajikan dalam bentuk tabel distribusi frekuensi dan tabel distribusi frekuensi relatif. Data : 1, 2, 3, 2, 3, 3, 4, 5, 3, 2, 3, 4, 5, 5, 5, 5, 3, 2, 1, 3.\\
\paragraph{Penyelesaian\\}
\includegraphics[width=\linewidth]{3}
Buat variabel \texttt{bilangan}, dan masukkan dengan fungsi \texttt{c}, dengan perintah\\
\texttt{>bilangan <- c(1, 2, 3, 2, 3, 3, 4, 5, 3, 2, 3, 4, 5, 5, 5, 5, 3, 2, 1, 3)\\}
cek banyaknya data pada variabel \texttt{bilangan} dengan perintah\\
\texttt{>length(bilangan)}\\
lalu buat tabel distribusi frekuensi dari variabel \texttt{bilangan} dengan perintah\\
\texttt{>table(bilangan)}
\subsubsection{Praktik 2}
\paragraph{Soal\\}
Sajikan data berikut dalam bentuk tabel distribusi frekuensi\\ 
1, 2, 3, 4, 5, 6, 7, 8, 9, 10, 10, 9, 8, 4, 3, 2\\ 
Kelas intervalnya 1 – 4, 5 - 10\\
\paragraph{Penyelesaian\\}
Buat variabel \texttt{bilangan} dan masukkan data menggunakan fungsi \texttt{c}\\
\texttt{>bilangan <- c(1, 2, 3, 4, 5, 6, 7, 8, 9, 10, 10, 9, 8, 4, 3, 2)}
\includegraphics[width=\linewidth]{4}
\section{Latihan}
\subsection{Latihan 1}
\paragraph{Soal\\}
Berikut ini data mahasiswa\\
\begin{table}[!ht]
	\begin{tabular}{|c|c|c|}
		\hline 
		nama & {gender} & {jurusan} \\ 
		\hline 
		Toni & Pria & D3 TI \\ 
		\hline 
		Tino & Pria & S1 SI \\ 
		\hline 
		Ana & Wanita & D3 MI \\ 
		\hline 
		Ina & Wanita & D3 TI \\ 
		\hline 
		Windha & Wanita & S1 TI \\ 
		\hline 
		Mega & Wanita & D3 MI \\ 
		\hline 
		Arif & Pria & S1 SI \\ 
		\hline 
		Tono &  Pria & D3 TI \\ 
		\hline 
		Linda & Wanita & D3 TI \\ 
		\hline 
		Paijo & Pria & S1 TI \\ 
		\hline 
	\end{tabular}
\end{table}\\
Tentukan:
\begin{enumerate}
	\item Tabel distribusi frekuensi mahasiswa menurut jurusannya
	\item Tabel distribusi frekuensi mahasiswa menurut gendernya
\end{enumerate}
\paragraph{Penyelesaian\\}
Pada latihan ini, saya memasukkan data menggunakan fungsi \texttt{read.table}, pertama buat tabel dalam text editor, dan diberi nama mahasiswa.txt.\\
\includegraphics[width=\linewidth]{mahasiswatxt}
Setelah itu masukkan tabel tersebut ke variabel \texttt{mahasiswa} beserta headernya\\
\texttt{>mahasiswa <- read.table("mahasiswa.txt",header=T)}.\\
\includegraphics[width=\linewidth]{5}
Untuk membuat tabel distribusi frekuensi mahasiswa menurut jurusan berarti kita akan memakai tabel jurusan dan mengambil data dari kolom jurusan, gunakan perintah\\
\texttt{>table(mahasiswa["jurusan"])}\\
Sedangkan untuk tabel distribusi frekuensi mahasiswa menurut gendernya berarti kita akan memakai tabel jurusan dan mengambil data dari kolom gender, gunakan perintah\\
\texttt{>table(mahasiswa["gender"])}\\
\newpage 
\subsection{Latihan 2}
\paragraph{Soal\\}
Sajikan data berikuta dalam tabel kontigensi
\begin{table}[!ht]
	\begin{tabular}{|l|l|l|l|}
		\hline
		\multicolumn{1}{|c|}{Jenis Kelamin} & \multicolumn{1}{c|}{Bidang} & \multicolumn{1}{c|}{Status} & \multicolumn{1}{c|}{Didik} \\ \hline
		Laki-laki                           & Marketing                   & Belum menikah               & SMU                        \\ \hline
		Perempuan                           & Marketing                   & Sudah menikah               & Sarjana                    \\ \hline
		Perempuan                           & Umum                        & Sudah menikah               & SMU                        \\ \hline
		Laki-laki                           & Akuntansi                   & Belum menikah               & Sarjana                    \\ \hline
		Perempuan                           & Marketing                   & Sudah menikah               & SMU                        \\ \hline
		Perempuan                           & Akuntansi                   & Sudah menikah               & Sarjana                    \\ \hline
		Perempuan                           & Akuntansi                   & Belum menikah               & Sarjana                    \\ \hline
		Laki-laki                           & Umum                        & Belum menikah               & Sarjana                    \\ \hline
		Perempuan                           & Marketing                   & Sudah menikah               & SMU                        \\ \hline
		Laki-laki                           & Marketing                   & Sudah menikah               & SMU                        \\ \hline
	\end{tabular}
\end{table}
\paragraph{Penyelesaian\\}
Buat tabel dalam text editor\\
\includegraphics[width=\linewidth]{pegawaitxt}
\includegraphics[width=\linewidth]{6}
masukkan tabel tadi ke dalam variabel \texttt{pegawai} dengan headernya menggunakan fungsi \texttt{read.table}\\
\texttt{>pegawai <- read.table("pegawai.txt",header=T)}\\
lalu buat tabel kontigensi dari variabel \texttt{pegawai}\\
\texttt{>ftable(pegawai)}
\newpage
\section{Tugas}
\subsection{Tugas 1}
\paragraph{Soal\\}
Sajikan data tersebut dalam bentuk tabel kontigensi
\begin{table}[!ht]
	\begin{tabular}{|l|l|}
		\hline
		\multicolumn{1}{|c|}{Daerah} & \multicolumn{1}{c|}{Barang} \\ \hline
		Bandung                      & Komputer                    \\ \hline
		Solo                         & TV                          \\ \hline
		Bandung                      & Komputer                    \\ \hline
		Bandung                      & TV                          \\ \hline
		Yogya                        & Radio                       \\ \hline
		Bandung                      & Komputer                    \\ \hline
		Solo                         & TV                          \\ \hline
		Solo                         & Radio                       \\ \hline
		Solo                         & Radio                       \\ \hline
		Bandung                      & TV                          \\ \hline
		Bandung                      & Komputer                    \\ \hline
		Solo                         & Radio                       \\ \hline
		Bandung                      & Radio                       \\ \hline
		Bandung                      & TV                          \\ \hline
	\end{tabular}
\end{table}
\paragraph{Penyelesaian\\}
\includegraphics[width=\linewidth]{tugas1txt}
Buat tabel ke dalam file .txt menggunakan text editor.\\
\includegraphics[width=\linewidth]{tugas1output}
Masukkan data dalam file .txt tadi ke dalam variabel menggunakan \texttt{read.table}.\\
\texttt{>tugas1 <- read.table("tugas1.txt",header=T)}\\
Buat tabel kontigensi dari tabel data tugas tadi dan memasukkannya ke variabel \texttt{tabeltugas1}\\
\texttt{>tabeltugas <- table(tugas1)}
\subsection{Tugas 2}
\paragraph{Soal\\}
Data hasil ujian akhir Statistika Elementer\\
23 60 79 32 57 74 52 70 82 36 80 77 81 95 41 65 92 85 55 76 10 64 75 78 25 98 67 71 83 54 72 88 62 43 89 84 48 90 15 34 17 69 63 61\\
Buatlah tabel distribusi frekuensi dengan kelas interval 20 – 39, 40 – 69, 70 - 100

\end{document}