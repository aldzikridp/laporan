% !TeX spellcheck = id_ID
\documentclass[a4paper,12pt]{article}
\usepackage[bahasa]{babel}
\usepackage{graphicx}
\usepackage{multirow}
\usepackage{enumitem}
\usepackage{listings}
\graphicspath{ {./img/} }
\begin{document}
\title{Laporan Praktikum Statistika Pertemuan 7}
%\author{Aldzikri Dwijayanto Prathama \\ {\small 195410189}}
\author{Aldzikri Dwijayanto Prathama 
	\\195410189}
\makeatletter
\begin{titlepage}
	\begin{center}
		{\huge \bfseries \@title }\\[14ex]
		\includegraphics[scale=.8]{logo}\\[4ex]
		{\large \@author}\\[20ex]
		{\large \bfseries {SEKOLAH TINGGI MANAJEMEN INFORMATIKA DAN KOMPUTER
				AKAKOM YOGYAKARTA}}
	\end{center}


%{\large \@date} 
\end{titlepage}
\makeatother
%\maketitle
\newpage
\section{Tujuan}
\begin{enumerate}
	\item Praktikan dapat melakukan penyajian data dalam bentuk tabel Kontingensi
	\item Praktikan dapat melakukan penyajian data dalam bentuk table distribusi Frekuensi
\end{enumerate}
\section{Dasar Teori}
\subsection{Tabel Kontigensi}
\paragraph{}
Tabel Kontigensi merupakan tabel yang digunakan untuk mengukur hubungan (asosiasi) antara dua variable kategorik dimana tabel tersebut merangkum frekuensi bersama dari observasi pada setiap kategori variable. 
\paragraph{}
Misalkan n sampel diklasifikasikan secara silang berdasarkan dua atribut atau lebih dalam suatu.
\paragraph{}
Berikut ini contoh tabel kontingesi 2x2 :
\begin{table}[!ht]
	\begin{tabular}{|c|c|l|l|c|}
		\hline
		\multicolumn{2}{|l|}{\multirow{2}{*}{}} & \multicolumn{2}{l|}{Variable 2}                 & \multirow{2}{*}{Total} \\ \cline{3-4}
		\multicolumn{2}{|l|}{}                  & \multicolumn{1}{c|}{1} & \multicolumn{1}{c|}{2} &                        \\ \hline
		\multirow{2}{*}{Variable 1}     & 1     & $O_{11}$                       &$O_{12}$                      &\multicolumn{1}{l|}{$n_{1+}$ }  \\ \cline{2-5} 
		& 2     & $O_{21}$                       & $O_{22}$                       & \multicolumn{1}{l|}{$n_{2+}$}  \\ \hline
		\multicolumn{2}{|c|}{Total}             & $n_{+1}$                       &  $n_{+2}$                      & N                      \\ \hline
	\end{tabular}
\end{table}
\paragraph{}
Dengan menggunakan R Console maka :
\begin{enumerate}
	\item Untuk membuat Tabel \textit{Contingency} dua arah dengan Fungsi \textit{\textbf{table( )}} dari  \textit{data.frame}
	\item Untuk membuat Tabel Contingency tiga Arah atau lebih dengan Fungsi \textit{\textbf{ftable( )}} untuk membentuk tabel contingency tiga arah dari \textit{data.frame}.
\end{enumerate}
\subsection{Distribusi Frekuensi}
\paragraph{}
Distribusi frekuensi adalah daftar nilai data (bisa nilai individual atau nilai data yang sudah dikelompokkan ke dalam selang interval tertentu) yang disertai dengan nilai frekuensi yang sesuai.
\paragraph{}
Pengelompokkan data ke dalam beberapa kelas dimaksudkan agar ciri-ciri penting data tersebut dapat segera terlihat. Daftar frekuensi ini akan memberikan gambaran yang khas tentang bagaimana keragaman data.
\paragraph{}
Distribusi frekuensi dibuat dengan alasan berikut:
\begin{enumerate}[label=\alph*.]
	\item kumpulan data yang besar dapat diringkas
	\item kita dapat memperoleh beberapa gambaran mengenai karakteristik data, dan
	\item merupakan dasar dalam pembuatan grafik penting (seperti histogram).
\end{enumerate}
\paragraph{}
Pada saat menyusun tabel distribusi frekuensi, pastikan bahwa 
\begin{enumerate}[label=\alph*.]
	\item kelas tidak tumpang tindih sehingga setiap nilai-nilai pengamatan harus masuk tepat ke dalam satu kelas
	\item tidak akan ada data pengamatan yang tertinggal (tidak dapat dimasukkan ke dalam kelas tertentu). 
\end{enumerate}
\paragraph{}
Dengan menggunakan R Console maka :
\begin{enumerate}
	\item Penyajian data dalam bentuk tabel  distribusi frequensi dapat digunakan \textbf{fungsi \textit{table()}}
	\item Untuk  penyajian data dalam bentuk tabel  distribusi frequensi relatif  digunakan \textbf{fungsi \textit{table()/length()}}
	\item Untuk Membuat Tabel Distribusi Frekuensi untuk Data Berkelompok digunakan \textbf{fungsi \textit{cut( )}} untuk membuat suatu interval. Argumen \textbf{\textit{break}} digunakan untuk menentukan batas-batas interval.	
\end{enumerate}
\section{Praktik}
\subsection{Tabel Kontigensi}
\subsubsection{Praktik 1}
\paragraph{}
Sajikan data berikut ini dalam bentuk table kontigensi 
\begin{table}[!ht]
	\begin{tabular}{|c|c|}
		\hline 
		Pendidikan & Jenis Kelamin \\ 
		\hline 
		S1 & Laki-laki \\ 
		\hline 
		S1 & Laki-laki \\ 
		\hline 
		S1 & Laki-laki \\ 
		\hline 
		S1 & Perempuan \\ 
		\hline 
		S1 & Perempuan \\ 
		\hline 
		S2 & Perempuan \\ 
		\hline 
		S2 & Perempuan \\ 
		\hline 
		S2 & Perempuan \\ 
		\hline 
		S2 & Perempuan \\ 
		\hline 
		S2 & Laki-laki \\ 
		\hline 
	\end{tabular}
\end{table} 
\includegraphics[width=\linewidth]{1}
\paragraph{}
Untuk memasukkan data tersebut, masukkan data dalam 
\subsubsection{Praktik 2}
\includegraphics[width=\linewidth]{2}
\subsection{Distribusi Frekuensi}
\subsubsection{Praktik 1}
\includegraphics[width=\linewidth]{3}
\subsubsection{Praktik 2}
\includegraphics[width=\linewidth]{4}
\section{Latihan}
\subsection{Latihan 1}
\includegraphics[width=\linewidth]{5}
\subsection{Latihan 2}
\includegraphics[width=\linewidth]{6}

\end{document}