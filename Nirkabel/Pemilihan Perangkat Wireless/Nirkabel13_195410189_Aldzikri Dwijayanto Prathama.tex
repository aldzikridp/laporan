% !TeX spellcheck = id_ID
\documentclass[a4paper,12pt]{article}
\usepackage[bahasa]{babel}
\usepackage{graphicx}
\usepackage{multirow}
\usepackage{enumitem}
\usepackage{listings}
\graphicspath{ {./img/} }
\begin{document}
\title{Memilih Perangkat Wireless}
%\author{Aldzikri Dwijayanto Prathama \\ {\small 195410189}}
\author{Aldzikri Dwijayanto Prathama
	\\195410189}
\makeatletter
\begin{titlepage}
	\begin{center}
		{\huge \bfseries \@title }\\[14ex]
		\includegraphics[scale=.8]{logo}\\[4ex]
		{\large \@author}\\[20ex]
		{\large \bfseries {SEKOLAH TINGGI MANAJEMEN INFORMATIKA DAN KOMPUTER \\
				AKAKOM YOGYAKARTA}}
	\end{center}


%{\large \@date} 
\end{titlepage}
\makeatother
%\maketitle
\newpage

\begin{center}
\section*{Memilih Perangkat Wireless}
\end{center}
\section{Jenis Sambungan}
\subsection{Wifi Indoor}
Wireless Indoor sesuai namanya memiliki pengertian sebuah perangkat wireless yang berfungsi untuk menghubungkan suatu
komputer/perangkat sejenis ke internet yang dipasang di dalam ruangan. Wireless indoor memiliki ciri cakupan yang cukup kecil
dan memiliki sinyal yang tidak terlalu kuat Untuk mengakses internet biasanya access point (atau dikenal dengan wifi
hotspot) dipasang di lokasi strategis dalam bangunnan yang memungkinkan orang menjangkau wifi hotspot tersebut dengan
mudah dan nyaman.

\subsection{Wireless Outdoor}
Pengertian wireless outdoor sendiri ialah perangkat wirelss yang dipasang di area terbuka dan luas dengan tujuan untuk
menyebarkan akses internet pada jangkauan yang luas. Jika wireless indoor dipasang di dalam bangunan atau dekat bangunan
maka wireless indoor memiliki ciri berbeda dengan pemasangannya yang berada di tempat tinggi, terbuka, dan menggunakan
perangkat tambahan seperti tower untuk pemasangannya.

\subsection{Perbedaan Perangkat Wireless Indoor dan Outdoor}
\subsubsection{Jarak}
Dikarenakan jarak jangkauan yang berbeda tentu alat yang digunakan berbeda dan memiliki bentuk yang berbeda. Wirelss Indoor
memiliki peralatan yang cukup sederhana (hanya aceess point dan kabel) dan tidak banyak dibandingkan wireless outdoor yang
memiliki bentuk yang berbeda dan diberi tambahan alat lainnya.

\subsubsection{Bangunan dan Alat Tambahan}
Wireless indoor yang hanya dipasang di dalam ruangan tidak memerlukan bangunan atau alat tambahan, hanya berupa kotak access
point dan kabel sudah cukup untuk mengakses internet. Sedangkan wireless outdoor memiliki perbedaan adanya tambahan bangunan
berupa tower dan alat seperti antena dan penangkal petir untuk dapat mengakses internet dengan aman dan nyaman.

\section{Wireless Indoor}
Hal yang harus diperhatikan saat memasang perangkat wireless Indoor
\begin{enumerate}
    \item \textbf{Perangkat Client\\}
        Perangkat client untuk jaringan nirkabel indoor cenderung memiliki kekuatan wireless transmite kecil.
        Contohnya:laptop, hp, tab, dll. Sehingga pemasangan indoor
        tidak memerlukan ukuran jangkauan sel yang besar. Cara untuk
        pemasangannya yaitu dengan cara meletakkan AP di setiap ruangan.

    \item \textbf{Jarak Sel\\}
            Jarak sel 10-15 meter tanpa adanya penghalang/ hambatan tembok,
            beton, dll.

    \item \textbf{Jumlah Client\\}
        Rata-rata jumlah client yang dapat terkoneksi dengan baik adalah 20 sampai 30 client, sehingga jika terdapat
        client lebih dari itu, maka kita memerlukan lebih banyak perangkat, dan dipasang secara tersebar.

    \item \textbf{Frekuensi\\}
        Frekuensi yang digunakan biasanya 2.4 GHz, tetapi perangk-perangkat baru biasanya sudah mendukung jaringan pada
        frekuensi 5 GHz, sehingga disarankan untuk menggunakan perengkat yang dapat memancarkan dua frekuensi sinyal
        yang berbeda secara bersamaan.

    \item \textbf{Peletakan\\}
        Untuk peletakan perangkat disarankan diletakan di tempat yang tinggi, sehingga jangkauan sinyal luas, dan
        hambatan lebih kecil. Selain itu peletekan diusahakan berada di tengah ruangan, sehingga seluruh ruangan
        terjangkau oleh sinyal perengkat.

    \item \textbf{Perangkat\\}
        Usahakan gunakan perengkat yang mendukung AC, atau dual frequency.\\
        Memakai perangkat wireless outdoor untuk indoor bisa dilakukan, tapi tidak disarankan. Karena ukurannya besar
        dan biasanya hanya mendukung satu frekuensi.\\
        Jangan menggunakan perangkat CPE untuk Access Point

    \item \textbf{Frekuensi\\}
        Beberapa perangkat dalam satu lokasi jangan menggunakan frekuensi yang sama\\
        TX Power disesuaikan dengan kebutuhan\\
\end{enumerate}

\section{Wireless Outdoor}
Hal yang harus diperhatikan untuk jaringan wireless outdoor.
\begin{enumerate}
   \item Topologi jaringan
   \item Kondisi ideal, jarak pandang dari titik satu ke titik sasaran tidak terhalang.
   \item Fresnel zone atau tempat merambatnya frekuensi ada.
   \item Tidak ada interferensi.
\end{enumerate}

\subsection{Software}
Software yang membantu dalam pemasangan jaringan wireless:
\begin{enumerate}
    \item Possibility Calculator
    \item Penghitung tinggi antenna
    \item Wireless Link Calculator
\end{enumerate}

\subsection{Topologi yang Digunakan}
\subsubsection{Point to Point}
digunakan untuk access point dengan memilih
perangkat yang lisensi minimalnya adalah level 3

\subsubsection{Point to Multipoint}
digunakan untuk access point dengan memilih
perangkat yang lisensi minimalnya adalah level 4

\subsection{Frekuensi}
\subsubsection{5 GHz}
Menggunakan parengkat dengan frekuensi 5 GHz lebih bersih dan baik digunakan untuk outdoor, karena lebih bersih,
bandwidth lebih besar, dan banyak perangkat baru yang mendukung frekuensi 5 GHz.\\
Contoh perangkat PtP yang mendukung frekuensi 5 Ghz:
\begin{enumerate}
    \item LDF 5 Series (biasa dan versi AC)
    \item SXT 5 Series
    \item SXTSQ 5 Series
\end{enumerate}

Contoh perangkat untuk jaringan menengah.
\begin{enumerate}
    \item SexTANT G
    \item QRT 5 Series
    \item Disc Lite 5 Series versi 5 GB AC
\end{enumerate}

Perangkat yang digunakan untuk jarak jauh
\begin{enumerate}
    \item LHG (HP 5, XL, 5 AC, XL 5 AC)
    \item Bynadish
\end{enumerate}

Perengkat point to multipoint sebagai access pointnya
\begin{enumerate}
   \item Omnitik 5GB versi AC atau non POE
   \item Omnitik versi POE atau omnitik versi AC
   \item SXT
\end{enumerate}

\subsubsection{2.4 GHz}
Penggunaan perengkat dengan frekuensi 2.4 GHz dianjurkan jika kompatibilitas menjadi kendala, karena perengkat lama
biasanya hanya mendukung frekuensi 2.4 GHz. Untuk jarak dekat pada frekuensi ini adalah 1 Km dan
dengan begitu perangkat yang biasanya digunakan adalah
perangkat LDF. Dengan frekuensi dengan ukuran ini dapat juga
dibuat dengan spesifikasi outdor.

Seri CPE yang sering digunakan:
\begin{enumerate}
    \item SXT (Square dan Classic), biasanya digunakan untuk jarak dekat
    \item QRT, digunakan untuk jarak menengah
    \item LHG, digunakan untuk jarak jauh
    \item LHG-XL, kita pilih yang memiliki frekuensi 2,4 GHz bukan yang 5 GHz
\end{enumerate}

Dari kedua frekuensi diatas, setiap antenanya di desain dengan
desain tertentu. Jika kita membuat access point to multipoint, maka
kita tidak bisa menggunakan frekuensi 2,4 GHz. Untuk pemasangan
outdor, baik menggunakan frekuensi 5 GHz atau 2,4 GHz hanya bisa
mengarah ke titik tertentu (directional).

\end{document}
