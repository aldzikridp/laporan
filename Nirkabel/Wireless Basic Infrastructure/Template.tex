% !TeX spellcheck = id_ID
\documentclass[a4paper,12pt]{article}
\usepackage[bahasa]{babel}
\usepackage{graphicx}
\usepackage{multirow}
\usepackage{enumitem}
\usepackage{listings}
\graphicspath{ {./img/} }
\begin{document}
\title{Dasar Infrastruktur Nirkabel dan Topologi}
%\author{Aldzikri Dwijayanto Prathama \\ {\small 195410189}}
\author{Aldzikri Dwijayanto Prathama
	\\195410189}
\makeatletter
\begin{titlepage}
	\begin{center}
		{\huge \bfseries \@title }\\[14ex]
		\includegraphics[scale=.8]{logo}\\[4ex]
		{\large \@author}\\[20ex]
		{\large \bfseries {SEKOLAH TINGGI MANAJEMEN INFORMATIKA DAN KOMPUTER
				AKAKOM YOGYAKARTA}}
	\end{center}


%{\large \@date} 
\end{titlepage}
\makeatother
%\maketitle

\section{Tujuan}
\begin{itemize}
    \item Mengetahui topologi jaringan nirkabel yang paling banyak digunakan
    \item Mampu mengidentifikasi dan merencanakan topologi yang cocok pada pada skenario nyata
    \item Memberikan pengenalan singkat untuk pemasangan praktis
\end{itemize}

\newpage

\tableofcontents

\newpage

\section{Topologi yang Relevan untuk Jaringan Nirkabel}
\begin{itemize}
    \item Star: Dapat diterapkan, topologi nirkabel yang standar
    \item Tree: Dapat diterapkan (kombinasi antara topologi star dan line)
    \item Line: Dapat diterapkan, dengan dua elemen atau lebih (PtP)
    \item Mesh: Dapat diterapkan, biasanya partial mesh
    \item Ring: Dapat diterapkan, namun jarang ditemui
    \item Bus: Tidak dapat diterapkan
\end{itemize}

\subsection{Beberapa Keterangan Umum}
\begin{itemize}
    \item Komunikasi nirkabel tidak membutuhkan medium
        \subitem Gelombang EM menyebar tanpa medium
        \subitem Garis pada diagram jaringan merupakan sambungan yang dibuat

    \item Komunikasi nirkabel selalu 2 arah
        \subitem Terkecuali untuk passive sniffing
        \subitem Berlaku untuk pemancar/penerima, client/master
\end{itemize}

\section{Komponen Jaringan Nirkabel}
\begin{itemize}
    \item Acces Point
        \subitem pemancar/penerima nirkabel yang menghubungkan antara node jaringan nirkabel, dan jaringan kabel
        \subitem IEEE 802.11 + Koneksi kabel Ethernet

    \item Client Nirkabel
        \subitem Semua komputer yang memiliki kartu adapter jaringan nirkabel yang memancarkan dan menerima sinyal RF
        \sub Laptop, PDA, perangkat pengintai, dan telepon VoIP
\end{itemize}

\section{Dasar Mode Nirkabel}
\subsection{Ad Hoc (Peer-to-Peer)}
\begin{itemize}
    \item Merupakan \textit{Independent Basic Service Set} (IBSS)
    \item Tidak membutuhkan access point pusat
    \item Semua node harus menggunakan SSID dan channel yang sama
    \item Tidak terskalakan
\end{itemize}

\subsection{Infrastructure}
\begin{itemize}
    \item Merupakan \textit{Extended Service Set} (ESS)
    \item Membutuhkan access point pusat
    \item "Menghubungkan" Wlan ke jaringan ethernet
    \item Terskalakan
\end{itemize}

\subsection{Keterangan Mengenai Mode Wireless}
\begin{itemize}
    \item Pada kedua mode, SSID mengidentifikasi jaringan
    \item Pertimbangkan SSID sebagai "label" dari Ethernet jack pada dinding
    \item Mode (mode dari operasi) bisa tersembunyi dan tidak terlihat di dalam topologi
        \subitem contoh: link PtP dapat berupa \textit{ad hoc} atau \textit{infrastructure}
\end{itemize}

\section{Contoh Kasus}
\subsection{Contoh Kasus Ad hoc 1: Point-to-Point}
\begin{itemize}
    \item Menghubungkan dua client secara langsung
    \item gedung satu ke gedung lain (ketika salah satu gedung memiliki akses internet sedangkan gedung lainnya tidak)
    \item Didalam kantor
\end{itemize}

\subsection{Contoh Kasus Infrastructure 1: Star}
\begin{itemize}
    \item Hotspot, Telecenter, KAntor, dan WISP's
    \item Point to Multi-point
    \item infrastruktur yang paling umum dalam jaringan nirkabel
\end{itemize}

\subsection{Contoh Kasus Infrastructure 2: Point-to-Point}
\begin{itemize}
    \item elemen dasar dari infrastruktur nirkabel 
    \item PtP link bisa bagian dari:
        \subitem star, tree, two point line atau topologi lain
\end{itemize}

\subsection{Contoh Kasus Infrastructure 3: Repeating}
\begin{itemize}
    \item Diperlukan ketika \textit{direct line of sight} (LOS) terhambat 
    \item repeating unit terdiri dari:
        \subitem satu atau dua perangkat fisik
        \subitem dua radio atau satu radio dan \textit{"isolated antenna"}
    \item Dapat dilihat sebagai client penerima dan akses point pemancar ulang
\end{itemize}

\subsection{Contoh Kasus Infrastructure 4: Mesh}
\begin{itemize}
    \item Topologi mesh adalah pilihan yang menarik terutama di
        \subitem Lingkungan yang dinamis (daerah perkotaan)
        \subitem Di mana infrastruktur pusat sulit diimplementasikan
        \subitem Ketika redundansi diinginkan
    \item Kasus umum adalah: jaringan pada kota, jaringan pada kampus dan Perumahan
    \item Topologi full mesh yaitu setiap node dapat terhubung ke node yang lain
    \item Topologi partial mesh yaitu setiap node dapat hanya dapat terhubung ke beberapa namun tidak ke semua node
    \item Pada sebuah mesh tidak harus dinamis
    \item Digunakan sebagai sinonim untuk jaringan adhoc atau mobile
    \item Semua mesh node harus berjalan dengan menggunakan mesh routing protocol yang sama
    \item Setiap node dapat menggunakan sistem operasi, dan tipe perangkat keras yang berbeda selama sesuai dengan
        spesifikasi protokol mesh.
\end{itemize}

\section{Contoh Infrastruktur Nirkabel Pada Kehidupan Nyata}
\begin{itemize}
    \item jaringan nirkabel pada kerhidupan nyata biasanya merupakan kombinasi lebih dari satu topologi
    \item Representasi grafis hanya sebagai contoh, dan berbeda satu sama lain.
\end{itemize}

\newpage

\section{Kesimpulan}
\begin{itemize}
    \item Kebanyakan implementasi jaringan nirkabel didasari oleh topologi star, tree, atau line.
    \item Pada implementasi kita bisa menemukan dua mode, Ad hoc atau infrastruktur (lebih sering ditemui)
    \item Dasar dari semua susunan terdiri dari:
        \subitem Mode, SSID, Channel + Mac/auth + IP
    \item Kebanyakan implementasi dari jaringan nirkabel terdiri dari lebih dari satu topologi
\end{itemize}

\end{document}
