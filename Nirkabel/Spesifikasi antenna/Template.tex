% !TeX spellcheck = id_ID
\documentclass[a4paper,12pt]{article}
\usepackage[bahasa]{babel}
\usepackage{graphicx}
\usepackage{multirow}
\usepackage{enumitem}
\usepackage{listings}
\graphicspath{ {./img/} }
\begin{document}
\title{Antenna dan Parameter Antenna}
%\author{Aldzikri Dwijayanto Prathama \\ {\small 195410189}}
\author{Aldzikri Dwijayanto Prathama 
	\\195410189}
\makeatletter
\begin{titlepage}
	\begin{center}
		{\huge \bfseries \@title }\\[14ex]
		\includegraphics[scale=.8]{logo}\\[4ex]
		{\large \@author}\\[20ex]
		{\large \bfseries {SEKOLAH TINGGI MANAJEMEN INFORMATIKA DAN KOMPUTER
				AKAKOM YOGYAKARTA}}
	\end{center}


%{\large \@date} 
\end{titlepage}
\makeatother
%\maketitle
\newpage
\section{Penjelasan}
Antenna-antenna yang dijual di pasaran, biasanya memiliki parameter yang dicantumkan pada deskripsi produk. Pada bab ini dijelaskan terlebih dahulu, parameter-
parameter yang biasanya dicantumkan pada deskripsi produk antenna.

\subsection{Gain}
Gain (Penguatan) bukanlah kuantitas yang bisa didefinisikan dalam bentuk fisik seperti Wattatau Ohm, tetapi Gain adalah rasio yang tidak berdimensi. Gain
diberikan sesuai dengan rujukan kepada antena standar. Dua antena yang biasanya digunakan sebagai rujukan adalah antena isotropic dan antena dipole
setengah gelombang. Antena Isotropic memancar sama baiknya ke segala arah. Antena isotropic yang sesungguhnya tidak pernah ada, tetapi antena ini menyediakan 
pola antena teoretis yang dan sederhana yang dapat dibandingkan dengan antena sesungguhnya. Antena mana pun yang sesungguhnya akan memancarkan lebih banyak energi
di beberapa arah daripada yang lainnya. Karena antena tidak bisa menciptakan energi, total data yang di pancarkan adalah sama dengan antena isotropic. Energi
tambahan apapun yang terpancar dalam arah yang dipilih akan diimbangi oleh pengurangan energi yang sama atau kurang di arah yang lain. Gain sebuah antena pada
sebuah arah adalah banyaknya energi yang dipancarkan dalam arah itu sebanding dengan energi yang diradiasikan oleh antena isotropic dalam arah yang sama ketika 
didorong dengan daya masukan yang sama. Biasanya kita hanya tertarik padagain maksimum, yang merupakan gain dalam arah dimana antena memancarkan sebagian besar
dayanya. Gain antena sebanyak 3 dB dibandingkan dengan antena isotropic akan ditulis sebagai 3 dBi.  Sebuah dipole separuh-gelombang yang beresonansi akan
menjadi standar yang berguna untuk dibandingkan dengan antena lain di satu frekuensi atau di lebarpita frekuensi yang sangat sempit. Untuk membandingkan dipole ke
sebuah antena padalebar frekuensi memerlukan sejumlah dipole dengan panjang yang berbeda. Gain antena sebanyak 3 dB dibandingkan dengan antena dipole akan ditulis
sebagai 3 dBd.

\subsection{Beamwidth}
Beamwidth  antenna biasanya dipahami sebagai lebar beam saat daya setengah. Puncak intensitas radiasi ditemukan, dan
lalu ujung kedua puncak yang melambangkan setengahdaya intensitas puncak ditemukan. Jarak bersiku di antara ke dua ujung
daya setengah didefinisikan sebagai beamwidth. Setengah daya yang diekspresikan dalam decible adalah-3dB, sehingga
beamwidth setengah daya beamwidth kadang-kadang dirujuk sebagai beamwidth 3dB. Beamwidth horisontal maupun
vertikal biasanya dipertimbangkan. Dengan asumsi bahwa sebagian besar daya yang dipancarkan tidak dibagi-bagi ke
dalamsidelobe,  gain kedepan akan berbanding terbalik dengan beamwidth: pada saat beamwidthberkurang, gain ke depan
bertambah.

\subsection{Polarisasi}
Polarisasi didefinisikan sebagai orientasi medan listrik gelombang elektromagnetik. Polarisasi pada umumnya
digambarkan seperti elips. Dua kasus istimewa polarisasi elips adalah polarisasi linear dan polarisasi sirkular. Awal
polarisasi gelombang radio ditentukan oleh antena.

\end{document}
