\documentclass{beamer}
\usetheme{Rochester}
\usepackage{tcolorbox}
\usepackage{soul}
\usepackage{graphicx}
\usepackage{multirow}
\usepackage{enumitem}
\usepackage{listings}
\usepackage{wrapfig}
\usepackage[T1]{fontenc}
\usepackage{inconsolata}
\usepackage{lipsum}
\usepackage{adjustbox}
\usepackage{upquote}


\usepackage{color}
\usepackage[table]{xcolor}
\definecolor{lightgray}{rgb}{0.95, 0.95, 0.95}
\definecolor{darkgray}{rgb}{0.4, 0.4, 0.4}
%\definecolor{purple}{rgb}{0.65, 0.12, 0.82}
\definecolor{editorGray}{rgb}{0.95, 0.95, 0.95}
\definecolor{editorOcher}{rgb}{1, 0.5, 0} % #FF7F00 -> rgb(239, 169, 0)
\definecolor{editorGreen}{rgb}{0, 0.5, 0} % #007C00 -> rgb(0, 124, 0)
\definecolor{orange}{rgb}{1,0.45,0.13}		
\definecolor{olive}{rgb}{0.17,0.59,0.20}
\definecolor{brown}{rgb}{0.69,0.31,0.31}
\definecolor{purple}{rgb}{0.38,0.18,0.81}
\definecolor{lightblue}{rgb}{0.1,0.57,0.7}
\definecolor{lightred}{rgb}{1,0.4,0.5}
% CSS
\lstdefinelanguage{CSS}{
  keywords={color,background-image:,margin,padding,font,weight,display,position,top,left,right,bottom,list,style,border,size,white,space,min,width, transition:, transform:, transition-property, transition-duration, transition-timing-function},	
  sensitive=true,
  morecomment=[l]{//},
  morecomment=[s]{/*}{*/},
  morestring=[b]',
  morestring=[b]",
  alsoletter={:},
  alsodigit={-}
}

% JavaScript
\lstdefinelanguage{JavaScript}{
  morekeywords={typeof, new, true, false, catch, function, return, null, catch, switch, var, if, in, while, do, else, case, break},
  morecomment=[s]{/*}{*/},
  morecomment=[l]//,
  morestring=[b]",
  morestring=[b]'
}

\lstdefinelanguage{HTML5}{
  language=html,
  sensitive=true,	
  alsoletter={<>=-},	
  morecomment=[s]{<!-}{-->},
  tag=[s],
  otherkeywords={
  % General
  >,
  % Standard tags
	<!DOCTYPE,
  </html, <html, <head, <title, </title, <style, </style, <link, </head, <meta, />,
	% body
	</body, <body,
	% Divs
	</div, <div, </div>, 
	% Paragraphs
	</p, <p, </p>,
	% scripts
	</script, <script,
  % More tags...
  <canvas, /canvas>, <svg, <rect, <animateTransform, </rect>, </svg>, <video, <source, <iframe, </iframe>, </video>, <image, </image>, <header, </header, <article, </article
  },
  ndkeywords={
  % General
  =,
  % HTML attributes
  charset=, src=, id=, width=, height=, style=, type=, rel=, href=,
  % SVG attributes
  fill=, attributeName=, begin=, dur=, from=, to=, poster=, controls=, x=, y=, repeatCount=, xlink:href=,
  % properties
  margin:, padding:, background-image:, border:, top:, left:, position:, width:, height:, margin-top:, margin-bottom:, font-size:, line-height:,
	% CSS3 properties
  transform:, -moz-transform:, -webkit-transform:,
  animation:, -webkit-animation:,
  transition:,  transition-duration:, transition-property:, transition-timing-function:,
  }
}

\lstdefinestyle{htmlcssjs} {%
  % General design
%  backgroundcolor=\color{editorGray},
  basicstyle={\footnotesize\ttfamily},   
  frame=single,
  % line-numbers
  % Code design
  identifierstyle=\color{black},
  keywordstyle=\color{blue}\bfseries,
  ndkeywordstyle=\color{editorGreen}\bfseries,
  stringstyle=\color{editorOcher}\ttfamily,
  commentstyle=\color{brown}\ttfamily,
  % Code
  language=HTML5,
  alsolanguage=JavaScript,
  alsodigit={.:;},	
  tabsize=2,
  showtabs=false,
  showspaces=false,
  showstringspaces=false,
  extendedchars=true,
  breaklines=true,
  % German umlauts
  literate=%
  {Ö}{{\"O}}1
  {Ä}{{\"A}}1
  {Ü}{{\"U}}1
  {ß}{{\ss}}1
  {ü}{{\"u}}1
  {ä}{{\"a}}1
  {ö}{{\"o}}1
}
%
\lstdefinestyle{py} {%
language=python,
literate=%
*{0}{{{\color{lightred}0}}}1
{1}{{{\color{lightred}1}}}1
{2}{{{\color{lightred}2}}}1
{3}{{{\color{lightred}3}}}1
{4}{{{\color{lightred}4}}}1
{5}{{{\color{lightred}5}}}1
{6}{{{\color{lightred}6}}}1
{7}{{{\color{lightred}7}}}1
{8}{{{\color{lightred}8}}}1
{9}{{{\color{lightred}9}}}1,
basicstyle=\footnotesize\ttfamily, % Standardschrift
numbers=left,               % Ort der Zeilennummern
%numberstyle=\tiny,          % Stil der Zeilennummern
%stepnumber=2,               % Abstand zwischen den Zeilennummern
numbersep=5pt,              % Abstand der Nummern zum Text
tabsize=4,                  % Groesse von Tabs
extendedchars=true,         %
breaklines=true,            % Zeilen werden Umgebrochen
keywordstyle=\color{blue}\bfseries,
frame=b,
commentstyle=\color{brown}\itshape,
stringstyle=\color{editorOcher}\ttfamily, % Farbe der String
showspaces=false,           % Leerzeichen anzeigen ?
showtabs=false,             % Tabs anzeigen ?
xleftmargin=17pt,
framexleftmargin=17pt,
framexrightmargin=5pt,
framexbottommargin=4pt,
%backgroundcolor=\color{lightgray},
showstringspaces=false,      % Leerzeichen in Strings anzeigen ?
}%
%
\definecolor{dkgreen}{rgb}{0,.6,0}
\definecolor{dkblue}{rgb}{0,0,.6}
\definecolor{dkyellow}{cmyk}{0,0,.8,.3}

\lstdefinestyle{PHP}{
  language        = php,
  basicstyle      = \small\ttfamily,
  keywordstyle    = \color{dkblue},
  stringstyle     = \color{red},
  identifierstyle = \color{dkgreen},
  commentstyle    = \color{gray},
  emph            =[1]{php},
  emphstyle       =[1]\color{black},
  emph            =[2]{if,and,or,else},
  emphstyle       =[2]\color{dkyellow}}
\lstset{
    showstringspaces=false,
    frame=single,
    breaklines=true,
    rulecolor=\color{black}
}
%
\graphicspath{ {./img/} }

\title{Menyisipkan File Multimedia ke HTML}
\author{Aldzikri Dwijayanto Prathama/195410189}

\begin{document}

\maketitle

\begin{frame}
    \frametitle{Apa Itu Multimedia?}
    Multimedia terdiri dari bermacam-macam format, yang dapat dilihat, atau didengarkan.\\
    Contohnya antara lain: gambar, music, suara, video, rekaman, film, animasi, dan lain-lain.\\
    Halaman web biasanya terdiri dari elemen multimedia yang beragam type dan formatnya.\\
\end{frame}

\begin{frame}
    \frametitle{Format Multimedia}
    Element multimedia seperti audio atau video disimpan dalam file media.\\
    cara paling umum untuk mengetahui type dari file, adalah dengan melihat extensi file.\\
    File multimedia memiliki format dan extensi yang berbeda, seperti:.swf,.wav,.mp3,.mp4,.mpg,.wmv,dan.avi.
\end{frame}

\begin{frame}
    \frametitle{Format Video dan Audio}
    Untuk video disarankan menggunakan format MP4, WebM, dan Ogg yang didukung oleh standard HTML 5\\
    Sedangkan untuk audio disarankan menggunakan MP3, WAV, dan Ogg.
\end{frame}

\begin{frame}
    \frametitle{Tag Untuk Menampilkan Mjltimedia}
    \begin{table}[!ht]
        \begin{tabular}{|l|l|}
            \hline
            \textless{}video\textgreater{}  & Menampilkan media berupa video                                                                                        \\ \hline
            \textless{}Audio\textgreater{}  & Menampilkan media berupa audio.                                                                                       \\ \hline
            \textless{}source\textgreater{} & \begin{tabular}[c]{@{}l@{}}Elemen untuk menentukan file media yang \\ dapat diputar dalam suatu browser.\end{tabular} \\ \hline
        \end{tabular}
    \end{table}
\end{frame}

\begin{frame}
    \frametitle{Tag Atribut Untuk Elemen Audio dan Video}
    \begin{table}[!ht]
        \begin{tabular}{|l|l|}
            \hline
            \textless{}autoplay\textgreater{} & Menentukan video/audio diputar secara otomatis. \\ \hline
            \textless{}width\textgreater{}    & Mengatur lebar dari video player.               \\ \hline
            \textless{}height\textgreater{}   & Mengatur tinggi dari video player.              \\ \hline
        \end{tabular}
    \end{table}
\end{frame}

\begin{frame}
    \frametitle{Implementasi}
    \begin{columns}
        \column{.5\linewidth}
        \includegraphics[width=\linewidth]{1.png}
        \column{.5\linewidth}
        \includegraphics[width=\linewidth]{2.png}
    \end{columns}
\end{frame}

\begin{frame}
    \frametitle{Implementasi}
    \begin{center}
        \includegraphics[scale = .2]{5.png} 
    \end{center}
    \begin{columns}
        \column{.5\linewidth}
        \begin{center}
        \includegraphics[scale = .2]{3.png}
        \end{center}
        \column{.5\linewidth}
        \begin{center}
        \includegraphics[scale = .2]{4.png}
        \end{center}
    \end{columns}
\end{frame}

\end{document}
