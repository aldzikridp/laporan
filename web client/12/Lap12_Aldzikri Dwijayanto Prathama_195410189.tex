\documentclass[a4paper,12pt]{article}
\usepackage[indonesian]{babel}
\usepackage{graphicx}
\usepackage{multirow}
\usepackage{enumitem}
\usepackage{listings}
\usepackage{wrapfig}
\usepackage[T1]{fontenc}
\usepackage{inconsolata}
\usepackage{lipsum}
\usepackage{adjustbox}
\usepackage{upquote}


\usepackage{color}
\usepackage[table]{xcolor}
\definecolor{lightgray}{rgb}{0.95, 0.95, 0.95}
\definecolor{darkgray}{rgb}{0.4, 0.4, 0.4}
%\definecolor{purple}{rgb}{0.65, 0.12, 0.82}
\definecolor{editorGray}{rgb}{0.95, 0.95, 0.95}
\definecolor{editorOcher}{rgb}{1, 0.5, 0} % #FF7F00 -> rgb(239, 169, 0)
\definecolor{editorGreen}{rgb}{0, 0.5, 0} % #007C00 -> rgb(0, 124, 0)
\definecolor{orange}{rgb}{1,0.45,0.13}		
\definecolor{olive}{rgb}{0.17,0.59,0.20}
\definecolor{brown}{rgb}{0.69,0.31,0.31}
\definecolor{purple}{rgb}{0.38,0.18,0.81}
\definecolor{lightblue}{rgb}{0.1,0.57,0.7}
\definecolor{lightred}{rgb}{1,0.4,0.5}
% CSS
\lstdefinelanguage{CSS}{
  keywords={color,background-image:,margin,padding,font,weight,display,position,top,left,right,bottom,list,style,border,size,white,space,min,width, transition:, transform:, transition-property, transition-duration, transition-timing-function},	
  sensitive=true,
  morecomment=[l]{//},
  morecomment=[s]{/*}{*/},
  morestring=[b]',
  morestring=[b]",
  alsoletter={:},
  alsodigit={-}
}

% JavaScript
\lstdefinelanguage{JavaScript}{
  morekeywords={typeof, new, true, false, catch, function, return, null, catch, switch, var, if, in, while, do, else, case, break},
  morecomment=[s]{/*}{*/},
  morecomment=[l]//,
  morestring=[b]",
  morestring=[b]'
}

\lstdefinelanguage{HTML5}{
  language=html,
  sensitive=true,	
  alsoletter={<>=-},	
  morecomment=[s]{<!-}{-->},
  tag=[s],
  otherkeywords={
  % General
  >,
  % Standard tags
	<!DOCTYPE,
  </html, <html, <head, <title, </title, <style, </style, <link, </head, <meta, />,
	% body
	</body, <body,
	% Divs
	</div, <div, </div>, 
	% Paragraphs
	</p, <p, </p>,
	% scripts
	</script, <script,
  % More tags...
  <canvas, /canvas>, <svg, <rect, <animateTransform, </rect>, </svg>, <video, <source, <iframe, </iframe>, </video>, <image, </image>, <header, </header, <article, </article
  },
  ndkeywords={
  % General
  =,
  % HTML attributes
  charset=, src=, id=, width=, height=, style=, type=, rel=, href=,
  % SVG attributes
  fill=, attributeName=, begin=, dur=, from=, to=, poster=, controls=, x=, y=, repeatCount=, xlink:href=,
  % properties
  margin:, padding:, background-image:, border:, top:, left:, position:, width:, height:, margin-top:, margin-bottom:, font-size:, line-height:,
	% CSS3 properties
  transform:, -moz-transform:, -webkit-transform:,
  animation:, -webkit-animation:,
  transition:,  transition-duration:, transition-property:, transition-timing-function:,
  }
}

\lstdefinestyle{htmlcssjs} {%
  % General design
%  backgroundcolor=\color{editorGray},
  basicstyle={\footnotesize\ttfamily},   
  frame=single,
  % line-numbers
  % Code design
  identifierstyle=\color{black},
  keywordstyle=\color{blue}\bfseries,
  ndkeywordstyle=\color{editorGreen}\bfseries,
  stringstyle=\color{editorOcher}\ttfamily,
  commentstyle=\color{brown}\ttfamily,
  % Code
  language=HTML5,
  alsolanguage=JavaScript,
  alsodigit={.:;},	
  tabsize=2,
  showtabs=false,
  showspaces=false,
  showstringspaces=false,
  extendedchars=true,
  breaklines=true,
  % German umlauts
  literate=%
  {Ö}{{\"O}}1
  {Ä}{{\"A}}1
  {Ü}{{\"U}}1
  {ß}{{\ss}}1
  {ü}{{\"u}}1
  {ä}{{\"a}}1
  {ö}{{\"o}}1
}
%
\lstdefinestyle{py} {%
language=python,
literate=%
*{0}{{{\color{lightred}0}}}1
{1}{{{\color{lightred}1}}}1
{2}{{{\color{lightred}2}}}1
{3}{{{\color{lightred}3}}}1
{4}{{{\color{lightred}4}}}1
{5}{{{\color{lightred}5}}}1
{6}{{{\color{lightred}6}}}1
{7}{{{\color{lightred}7}}}1
{8}{{{\color{lightred}8}}}1
{9}{{{\color{lightred}9}}}1,
basicstyle=\footnotesize\ttfamily, % Standardschrift
numbers=left,               % Ort der Zeilennummern
%numberstyle=\tiny,          % Stil der Zeilennummern
%stepnumber=2,               % Abstand zwischen den Zeilennummern
numbersep=5pt,              % Abstand der Nummern zum Text
tabsize=4,                  % Groesse von Tabs
extendedchars=true,         %
breaklines=true,            % Zeilen werden Umgebrochen
keywordstyle=\color{blue}\bfseries,
frame=b,
commentstyle=\color{brown}\itshape,
stringstyle=\color{editorOcher}\ttfamily, % Farbe der String
showspaces=false,           % Leerzeichen anzeigen ?
showtabs=false,             % Tabs anzeigen ?
xleftmargin=17pt,
framexleftmargin=17pt,
framexrightmargin=5pt,
framexbottommargin=4pt,
%backgroundcolor=\color{lightgray},
showstringspaces=false,      % Leerzeichen in Strings anzeigen ?
}%
%
\definecolor{dkgreen}{rgb}{0,.6,0}
\definecolor{dkblue}{rgb}{0,0,.6}
\definecolor{dkyellow}{cmyk}{0,0,.8,.3}

\lstdefinestyle{PHP}{
  language        = php,
  basicstyle      = \small\ttfamily,
  keywordstyle    = \color{dkblue},
  stringstyle     = \color{red},
  identifierstyle = \color{dkgreen},
  commentstyle    = \color{gray},
  emph            =[1]{php},
  emphstyle       =[1]\color{black},
  emph            =[2]{if,and,or,else},
  emphstyle       =[2]\color{dkyellow}}
\lstset{
    showstringspaces=false,
    frame=single,
    breaklines=true,
    rulecolor=\color{black},
    style=htmlcssjs,
    numbers=left
}
%

\graphicspath{ {./img/} }
\begin{document}
\title{ {\Large Laporan Praktikum}\\ Pemrograman Web Client\\{\Large Pertemuan 12}}

\author{Aldzikri Dwijayanto Prathama 
	\\195410189
	\\Informatika}
\makeatletter
\begin{titlepage}
	\begin{center}
		{\huge \bfseries \@title }\\[14ex]
		\includegraphics[scale=.8]{logo}\\[4ex]
		{\large \@author}\\[12ex]
		{\large \bfseries {SEKOLAH TINGGI MANAJEMEN INFORMATIKA DAN KOMPUTER
				AKAKOM YOGYAKARTA}}
	\end{center}


%{\large \@date} 
\end{titlepage}
\makeatother
%\maketitle
\renewcommand{\figurename}{Gambar}
\newpage
\tableofcontents
\newpage
\section{Tujuan}
\begin{enumerate}
   \item Memformat konten teks dengan CSS (warna, bentuk dan jenis huruf, latar belakang)
   \item Positioning konten
\end{enumerate}
\newpage

\section{Pembahasan}
\subsection{Praktik}
\begin{lstlisting}
header, section, footer, aside, nav, article, figure, figcaption {
display : block;
}

body {
color: #666666;
background-color: #f9f8f6;
background-image: url("latar.jpg");
background-position: center;
font-family: Georgia, Times, serif;
line-height: 1.4em;
margin: 0px;
}

.wrapper {
width: 940px;
margin: 20px auto 20px auto;
border: 2px solid #000000;
background-color: #ffffff;
}

header{
height: 160px;
background-image: url("header1.png");
}

h1{
text-indent: -9999px;
width: 940px;
height: 130px;
margin: Opx;
}

nav, footer{
clear: both;
color: #ffffff;
background-color: #aeaca8;
height: 30px;
}

nav ul{
margin: Opx;
padding: 5px Opx 5px 30px;
}

nav li{
display: inline;
margin-left: 40px;
}

nav li a{
color: #ffffff;
}

nav li a:hover, nav li a.current {
color: #000000;
}

section.utama {
float: left;
width: 659px;
border-right: 1px solid #eeeeee;
}

arcicle {
clear: both;
overflow: auto;
width: 100%;
}

hgroup {
margin-top: 40px;
}

figure {
float: left;
width: 290px;
margin: 20px;
border: 1px solid #eeeeee;
}

figcaption{
font-size: 90%;
text-align: left;
}

aside {
width: 230px;
float: left;
padding: 0px 0px 0px 20px;
}

aside section a {
display: block;
padding: 10px;
border-bottom: 1px solid #eeeeee;
}

aside section a:hover{
color: #985d6a;
background-color: #efefef;
}

a{
color: #de6581;
text-decoration: none;
}

h1, h2, h3 {
font-weight: normal
}

h2 {
margin: 10px 0px 5px 0px;
padding: 0px;
}

h3 {
margin: 0px 0px 10px 0px;
color: #de6581;
}

aside h2 {
padding: 30px 0px 10px 0px;
color: #de6581;
}

footer {
font-size: 80%;
padding: 7px 0px 0px 20px;
}
\end{lstlisting}
Pada file CSS tersebut terdapat beberapa tiga class, yaitu .wrapper, .current, dan .utama.\\

Untuk elemen yang diatur oleh CSS tersebut antara lain
\begin{enumerate}
   \item Header
   \item Section
   \item Footer
   \item Aside
   \item Nav
   \item Article
   \item Figure
   \item Figcaption
   \item h1, h2, dan h3
\end{enumerate}

Sedangkan properties yang diatur pada CSS tersebut sebagai berikut:
\begin{enumerate}
    \item dislpay mengatur tampilan suatu elemen
    \item color mengatur warna
    \item background-color mengatur warna background
    \item Background-position digunakan untuk mengatur posisi dari background gambar tersebut.
    \item Font-family digunakan untuk mengatur style pada tulisan tersebut.
    \item Line-height digunakan untuk mengatur jarak antara paragraph di dalam sebuah halaman website.
    \item Margin digunakan untuk mengaplikasikan jarak tepi pada sebuah elemen bagian luar (tepi garis luar).
    \item Width digunakan untuk mengatur lebar dari sebuah element.
    \item Border digunkaan untuk membuat garis tepi sebuah objek html.
    \item Height digunakan untuk mengatur tinggi dari sebuah element.
    \item Text-indent digunakan untuk membuat baris pertama pada paragraph/kalimat menjorok kedalam.
    \item Clear digunakan untuk membersihkan element yang sebelumnya telah mengalami float. Sehingga, element yang di-clear tersebut akan berpisah dari element yang mengalami floar dan akan berpindah pda baris baru.
    \item Display digunakan untuk mengatur tampilan suatu element.
    \item Padding digunakan untuk membuat spasi/ruang diantara sebuah konten utama dan border.
    \item Margin-right digunakan untuk mengaplikasikan jarak tepi ke kanan pada sebuah element.
    \item Float digunakan untuk menentukan apakah sebuah element box harus mengapung atay tidak. Maksudnya, sebuah element bisa diposisikan seakan-akan berada mengapung diantara element setelahnya.
    \item Overflow digunakan untuk menyembunyikan, menampakkan, atau membuat scroll dan pada property ini, dapat diterapkan dalam objek teks ataupun gambar.
    \item Font-size digunakan untuk mengatur ukuran pada tulisan pada teks tersebut.
    \item Border-bottom digunakan untuk mengatur garis yang terletak pada bagian bawah.
    \item Text-decoration digunakan untuk memberikan efek garis pada suatu text.
\end{enumerate}

\newpage

\section{Kesimpulan}
Setelah praktik mahasiswa mampu memformat konten teks dengan CSS (warna, bentuk dan jenis huruf, latar belakang) dan positioning konten.

\end{document}
